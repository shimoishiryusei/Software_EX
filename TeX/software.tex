\documentclass[a4paper,11pt]{jsarticle}

% パッケージ
\usepackage[dvipdfmx]{hyperref}
\usepackage{pxjahyper}
\usepackage[dvipdfmx]{graphicx}
\usepackage{ascmac}
\usepackage{fancybox}
\usepackage{listings}
\usepackage{plistings}
\usepackage{color}
\usepackage{here}
\usepackage{amsmath,amsfonts}
\usepackage{bm}
\usepackage{siunitx}
\usepackage{url}
% ページの周りの余白
\usepackage[top=10truemm,bottom=10truemm,left=25truemm,right=25truemm]{geometry}
% ページ番号の削除
\pagestyle{empty}

% URLの設定
\Urlmuskip=0mu  plus 10mu

% 色の定義
\definecolor{OliveGreen}{rgb}{0.0,0.6,0.0}
\definecolor{Magenta}{cmyk}{0, 1, 0, 0}
\definecolor{colFunc}{rgb}{1,0.07,0.54}
\definecolor{CadetBlue}{cmyk}{0.62,0.57,0.23,0}
\definecolor{Brown}{cmyk}{0,0.81,1,0.60}
\definecolor{colID}{rgb}{0.63,0.44,0}

% ソースコードの設定
\lstset{
  classoffset = 0,
  keepspaces=true,
  basicstyle={\footnotesize},
  showstringspaces={false},
  identifierstyle={\small},
  commentstyle={\smallitshape},
  keywordstyle={\bfseries \color[cmyk]{0,1,0,0}},
  ndkeywordstyle={\small},
  stringstyle={\ttfamily \color[rgb]{0,0,1}},
  frame={tb},
  breaklines=true,
  breakindent = 10pt,
  columns=[l]{fullflexible},
  numbers=left,
  xrightmargin=0zw,
  xleftmargin=3zw,
  numberstyle={\scriptsize},
  stepnumber=1,
  numbersep=1zw,
  lineskip=-0.5ex
}

\renewcommand{\lstlistingname}{Code}

% リンクの設定
\hypersetup{
  setpagesize=false,
  bookmarksnumbered=true,
  bookmarksopen=true,
  colorlinks=true,
  linkcolor=blue,
  citecolor=red,
}

\begin{document}

\title{CI4 ソフトウェア実験}
\author{CI4 21番 下石 龍生}
\date{\today}
\maketitle


\section{目的}
本実験は,Script言語であるPythonの基礎を身につけることを目的とする.まず,ファイルの入出力を習得し,今後の研究,実験で活用できるようにする.

\section{Pythonとは}
Python\cite{python}とは分散オレペーティングシステム「Amoeba」のシステム管理を行うためにGuido van Rosseum氏が開発したプログラミング言語である.
インタプリタ型言語であり,読みやすく,エラーの修正がしやすいため,初心者でも扱いやすい言語である.
Pythonが活用されている例を以下に示す.
\begin{itemize}
  \item データの収集,処理,分析
  \item クローリング,スクレイピング
  \item 機械学習,ディープラーニング
  \item ゲーム,スマホアプリの開発
  \item Webアプリケーション開発
\end{itemize}


\section{実験環境}
実験環境を以下の表\ref{em}に示す.
\begin{table}[H]
  \begin{center}
    \caption{実験環境}
    \begin{tabular}{|c|c|c|}  \hline 
      デバイス &  OS & ソフト \\ \hline 
      MacBookAir2019 13inch &  MacOS BigSur 11.2.3 & Python3.9.4 \\ \hline
    \end{tabular}
    \label{em}
  \end{center}
\end{table}

\section{課題}
\subsection{データセット}
  本実験で使用したデータを以下のCode\ref{input.csv},\ref{input2.csv},\ref{data1.csv},\ref{data2.csv},\ref{data3.csv}に示す.
  \lstinputlisting[caption=input.csv, label=input.csv]{Python/input.csv}
  \lstinputlisting[caption=input2.csv, label=input2.csv]{Python/input2.csv}
  \lstinputlisting[caption=data1.csv, label=data1.csv]{Python/data1.csv}
  \lstinputlisting[caption=data2.csv, label=data2.csv]{Python/data2.csv}
  \lstinputlisting[caption=data3.csv, label=data3.csv]{Python/data3.csv}

\subsection{演習課題1}
  \begin{screen}
  30以下の偶数の2乗の和を計算・表示するscriptを作成せよ.
  \end{screen}
  作成したコードを以下のCode\ref{task1}に,その実行結果を図\ref{task1ans}に示す.
  \lstinputlisting[caption = 演習課題1, label = task1]{Python/task1.py}
  \begin{figure}[H]
    \centering 
    \includegraphics[width=0.8\linewidth]{Experiment_photo/task1.png}
    \caption{演習課題1の実行結果}
    \label{task1ans}
  \end{figure}
  演習課題1では,for文で0~31まで,2つ間隔で値をiに代入しているため偶数だけを取り除くことができる
  ようにしている.取り出した偶数iは,二乗して変数ansに足して更新するため,結果的に30以下の偶数の二乗和が
  計算されている.

\subsection{演習課題2}
  \begin{screen}
  input.csvを読み込んで.図12のように,奇数行の場合は行数および2列目+3列目を浮動小数型で,偶数行
  の場合は行数および4列目を文字型で,画面に表示するscriptを作成せよ.
  \end{screen}
  作成したコードをCode\ref{task2}に,その実行結果を図\ref{task2ans}に示す.なお,Code\ref{input.csv}を
  使用した.
  \lstinputlisting[caption = 演習課題2, label = task2]{Python/task2.py}
  \begin{figure}[H]
    \centering
    \includegraphics[width=0.8\linewidth]{Experiment_photo/task2.png}
    \caption{演習課題2の実行結果}
    \label{task2ans}
  \end{figure}
  演習課題2では,input.csvを1行ごと取り出してそれをさらに列ごとに配列の要素化している.配列にするときは
  ","区切りで分けた.行番号を2で割ったときに割り切れたときは偶数行,割り切れなかったときは奇数行として
  表示する形式を変更できるようにしている.

\subsection{演習課題3}
  \begin{screen}
    まずdata1.csvを保存し,data1.csvの1,2行目のコメント行をEditorなどで削除しなさい.
    このコメント行を削除したデータを読み込み,その3,8,10行目のみ,各行の1,2列目を表示するscriptを作成せよ.
    (Script例リスト7を参考にして,コメント行を削除せずに,その行を読み飛ばす処理を行っても良い.)
    さらに出力行を図13のようなリストで指定し,指定された行の1,2列目を表示するscriptを作成せよ.
    ただしリストloutputの要素数は3とは限らない.またlenを用いるとリストの要素数を取得することができる.
  \end{screen}
  この課題で作成したコードとその実行結果をそれぞれCode\ref{task3}と図\ref{task3ans}に示す.なお,Code\ref{data1.csv}
  を使用した.
  \lstinputlisting[caption=演習課題3, label=task3]{Python/task3.py}
  \begin{figure}[H]
    \centering
    \includegraphics[width=0.8\linewidth]{Experiment_photo/task3.png}
    \caption{演習課題3の実行結果}
    \label{task3ans}
  \end{figure}

\subsection{演習課題4}
  \begin{screen}
    前の例題で用いたcsvファイルdata1.csvおよびそれと同じフォーマットのcsvファイルdata2.csv,csvファイルdata3.csvのそれぞれに対して,
    温度(1列目),抵抗率(2列目),電気抵抗(7列目),電気抵抗の測定誤差(8列目)を抜き出し,ファイルに出力するscriptを作成せよ.ただし,
    抵抗値が負のデータに関しては,先頭に"\#\#"を付けて出力すること.また出力ファイル名は適切に設定すること.
  \end{screen}
  この課題で作成したコードをCode\ref{task4}に示す.また,実行結果をCode\ref{d1.t},\ref{d2.t},\ref{d2.t}に示す.
  なお,Code\ref{data1.csv},\ref{data2.csv},\ref{data3.csv}を使用した.
  \lstinputlisting[caption=演習課題4, label=task4]{Python/task4.py}
  \lstinputlisting[caption=出力1, label=d1.t]{Python/data1.txt}
  \lstinputlisting[caption=出力2, label=d2.t]{Python/data2.txt}
  \lstinputlisting[caption=出力3, label=d3.t]{Python/data3.txt}

\subsection{研究課題}
  \begin{screen}
    data1.csv,data2.csv,data3.csvに対して,以下の条件を満たすようなscriptを作成せよ.
    \begin{itemize}
      \item 課題4のscriptで,matplotlibを使ってグラフを描画する.
      \item 横軸: 温度、縦軸: 電気抵抗値.
      \item 抵抗値の最大値・最小値に応じて、自動的に縦軸の範囲を指定.
      \item 最大値と最小値が二桁以上異なる場合は縦軸を対数グラフ.
      \item 画像(pngフォーマット)に保存.
    \end{itemize}
  \end{screen}
  研究課題で作成したコードをCode\ref{ex1}に示す.また,実行結果を図\ref{ex1p},\ref{ex2p},\ref{ex3p}にそれぞれ示す.
  なお,Code\ref{data1.csv},\ref{data2.csv},\ref{data3.csv}を使用した.
  \lstinputlisting[caption=研究課題, label=ex1]{Python/ex1.py}
  \begin{figure}[H]
    \centering
    \includegraphics[width=0.8\linewidth]{Python/data1.csv.png}
    \caption{data1.csvのグラフ}
    \label{ex1p}
  \end{figure}
  \begin{figure}[H]
    \centering
    \includegraphics[width=0.8\linewidth]{Python/data2.csv.png}
    \caption{data2.csvのグラフ}
    \label{ex2p}
  \end{figure}
  \begin{figure}[H]
    \centering
    \includegraphics[width=0.8\linewidth]{Python/data3.csv.png}
    \caption{data3.csvのグラフ}
    \label{ex3p}
  \end{figure}

\section{考察}
  今回の実験の課題4にて,読み込むファイルと書き出すファイルを対応付けてfor文で繰り返すプログラムを作成した.このとき使用したzip関数について考察を行う.\cite{zip}\\
  まず,2つの配列を対応付けてfor文を作る際に考えられるアルゴリズムを以下に示す.
  \lstinputlisting{test.py}
  zip関数を利用することにより,複数のリストを同時に所得し処理を行うことができるようになるため,他のアルゴリズムよりもシンプルに書くことができた.
  zip関数で注意しなければ注意しなければならない点は,取得した2つ以上のリストで要素数が異なるときは,要素数の少ないほうに合わせて処理されることである.

  \begin{thebibliography}{1}
    \bibitem{zip} Python "標準ライブラリ" (最終閲覧日 2021年5月7日)\\ \url{https://docs.python.org/ja/3/library/functions.html}\\
    \bibitem{python} @IT "Pythonってどんな言語なの?" (最終閲覧日 2021年5月7日)\\ \url{https://www.atmarkit.co.jp/ait/articles/1904/02/news024.html}\\
  \end{thebibliography}




\end{document}